\documentclass[12pt,a4paper]{article}
\usepackage[utf8]{inputenc}
\usepackage[english]{babel}
\usepackage[bibnewpage,nosectionbib]{apacite}
\usepackage{longtable}
\usepackage{graphicx}
\usepackage{tikz}
\usetikzlibrary{shapes.geometric,arrows,positioning}
\author{Prodezzargenta}
\title{Cryptocurrencies, and open source brands}
\date{June 7, 2022}
\begin{document}

\maketitle

\begin{abstract}
Entrepreneurship proposal of an \textit{alternative} business model, in order to create parallel markets by manufacturing goods valued in cryptocurrencies, and to be independent of \textit{fiat} system.
\end{abstract}
	
\tableofcontents

\section{Introduction}
The creation of cryptocurrency has represented, in a higher or lower degree, a rupture in the economy's conception. Since the creation of the State, the economies were progressively administrated by these entities; and through history, it's evidenced how States legitimized the currency monopoly.

According to the article \textit{On the Origins of Money}, Carl Menger says:

\begin{quotation}
To assume that certain commodities, the precious metals in particular, had been exalted into the medium of exchange by general convention or law, in the interest of commonweal, solved the difficulty, and solved it apparently the more easily and naturally inasmuch as the shape of the coins seemed to be a token of state regulation. such in fact is the opinion of Plato, Aristotle, and the roman jurists, closely followed by the medieval writers.  \cite[p. 16]{menger:origins}
\end{quotation}

Giving the fact that money \textit{is and only is} a creation of the State, as a medium of exchange of different goods and services, is absurd. The economic experience about the direct exchange (barter) and its evolution towards indirect exchange (money) is recorded since the times where humanity discovered that metals such as copper, bronze, silver and gold can be treasured and survive the passage of time. Even more, as explained by Ludwig von Mises in his book \textit{The Human Action}:

\begin{quotation}
There were authors who tried to explain the origin of money by decree or covenant. The authority, the state, or a compact between citizens has purposely and consciously established indirect exchange and money. The main deficiency of this doctrine is not to be seen in the assumption that people of an age unfamiliar with indirect exchange and money could design a plan of a new economic order, entirely different from the real conditions of their own age, and could comprehend the importance of such a plan. Neither is it to be seen in the fact that history does not afford a clue for the support of such statements. There are more substantial reasons for rejecting it.

If it is assumed that the conditions of the parties concerned are improved by every step that leads from direct exchange to indirect exchange [...] it is difficult to conceive why one should, in dealing with the origin of indirect exchange, resort in addition to authoritarian decree or an explicit compact between citizens. [...] Under these circumstances there was no need of government interference or of a compact between the citizens. [...] It is certainly more plausible to take for granted that the immediate advantages conferred by indirect exchange were recognized by the acting parties than to assume that the whole image of a society trading by means of money was conceived by a genius and, if we adopt the covenant doctrine, made obvious to the rest of the people by persuasion. \cite[pp. 402-403]{mises:ha}
\end{quotation}

In summary, money is a creation of the same interaction between human being through history, in which they discover by themselves greater advantages than \textit{barter} (or direct exchange) to be able to trade.

The State, in this context, take possession of money administration and, then, the coinage. This situation, added up to the \textit{legitimized} monopoly of violence over a territory, leads up to a situation in which the incentives are not favorable for free trading. This means, on one hand, it's the State the \textit{only} entity, legitimized by itself, to administrate the coinage of money; and, on the other hand, is the entity that enforce violence on a territory.

And so, through the millennia, States has increased their power violently\footnote{At the moment of writing this article, there is an ongoing invasion of Russia on Ukraine, by reasons that has been changed significantly through these months. And this invasion suppose, also, an occupation of the Ukrainian territory to be annexed on Russia.}, both directly (Armed Forces) and indirectly (money). Through the application of taxes\footnote{The word “tax” in Spanish easily reveals the nature of its violent nature: “Impuesto”, that literally means “Imposed”.}, this entity has been able to survive the passage of time; and whoever who don't pay taxes in a timely manner, will be sanctioned in any way.

In the last centuries, when the State began to administrate education over the territory, the theoretical content has been reformed to justify its own existence, financing, among other things, theories that validates its violent acts. And that's how, for example, in countries like Argentina are, still, teaching that macroeconomic imbalances and the depreciation of the local currency (to name some of the actual problems) are purely and exclusively because of the private sector, and not the economic policies created by the State itself. Not only that, but also it's taught this fundamental idea in \textit{every} educational order (from preschool to universities), denying any already checked empirical refutation. This demonstrates that the main \textit{incentives} of the State is to increment its power; and to do this, it must exercise, progressively, more violence against people.

Since the launch of Bitcoin by Satoshi Nakamoto in the year 2009, despite the technical highs and lows that had, the nature of its \textit{open source} code has demonstrated to be a successful implementation for people around the world to contribute, voluntarily, to re-write the code, and improving the original protocol; or by creating new cryptocurrencies based on the same principle. And given the fact that uses cryptographic systems and security protocols, cryptocurrencies (as its name indicates) has been able to establish as candidates to be a medium of exchange of goods and services \textit{outside} the State control. In fact, there are a few \textit{fiat} currencies with the blockchain technology with the end of replacing the actual \textit{fiat} money in a “2.0 currency”.

\subsection{Actual situation}
Despite the creation of many cryptocurrencies, where there's currently registered 717 cryptocurrencies \cite{mps}, there's a \textit{general} tendency to conceive cryptocurrencies as a financial asset to earn more \textit{fiat} money. This implies that these are not considered as a medium of exchange of goods and services \textit{outside} the State control; but rather a mere financial asset to buy it “at low” and sell it “at high”, based upon arbitrary and subjective criteria, or in a sort of technical analysis of its price.

This change of conception can be justified by the multiple ads in the media and social network, based on the navigation history, interaction with other websites, etcetera; \textit{youtubers} financial recommendations (although some of them are not specialized in such discipline); or in the news about how different people “becoming millionaires” or “how they lost all of their money”. However, one cannot ignore the fact that there's not enough economic theory specifically about cryptocurrencies to offer a more detailed lecture of the phenomenon, and also there's an emotional impulse of introducing to the crypto world to earn more local money (or dollars, euros, etcetera) in the fastest way possible.

Beyond the multiple interpretations that it might be written about this (from the user's ignorance until those conspiracy theories, so far, about the State incentive for users to loose all of their money, and emerging as a provider of “financial security” in exchange of a lesser individual freedom), the truth is there's a \textit{misguided} conception about cryptocurrencies. Developers, also, are not extent of this problem. In Monero, for example, the most important development projects are decentralized exchanges (although there is some kind of controversy, yet in discussion, about the internal protocols). This might be justified with the argument of providing even more liquidity in the market; but there are decentralized alternatives already to exchange \textit{fiat} for Monero.

The point in this last statement refers to the central issue of this document: stop perceiving cryptocurrencies as a mere financial asset to acquire it when the exchange rate to \textit{fiat} is “low”, and sell partially or totally the crypto amount acquired when the price is “high” (and, thus, earning more \textit{fiat} money). This way, the human action will tend to see cryptocurrencies as a medium of exchange outside regulation and State control, and produce goods and services under a \textit{parallel} market.

\subsection{Legal issues}
This might be a controversial subject to discuss about the “crypto nature”. Without going into details on this, given the market incentives, the \textit{general} criteria about cryptocurrencies is the following: any crypto denomination being “accepted” by the States implies that these entities can already control indirectly people's purchasing power (whether being partially or totally)

The \textit{legal} exchanges are controlled by States. This means they have to inform about its patrimonial situation and their clients. For example, in the case of Bitcoin, despite the registered transaction in the blockchain of being pseudo-anonymous (\textit{i.e.} there's no registry of the legal data of the user such as name and last name), given certain condition, one can acquire such data. The exchange business model is the following: a financial entity, a third-party, that brings new clients to use their platform; and these entities provides with a private address to offer liquidity and ease of exchange between cryptocurrencies and \textit{fiat} money (in exchange, of course, of a fee by using its services)\footnote{Otherwise, people will have to physically find other persons willing to make such transaction.}. Therefore, this businesses will be in charge of treasure many cryptocurrencies to be able to be rapidly exchanged by its users. However, for users to start trading, it's a necessary requisite for clients to provide their legal data. Binance, the largest international exchange in the crypto world, requires the following data to take away any restriction to its customers for using their services:

\begin{itemize}
\item Country of residence, and nationality.
\item Name and last name, and birth date.
\item Address, city, and postal code.
\item Scanned image of the front and back of the ID, passport or driving licence.
\item Front image of the user for facial recognition.
\item Data verification (several days after the first data entry).
\item Scanned image of any service that proves the place of residence.
\end{itemize}

Taking away the worst-case scenario where the exchange's database is stolen and freely published on the internet, if the exchanges are legal entities that are regulated by the State, what prevents the State to obtain such data, given an arbitrarily situation? If the consequences of not following the State's orders is the permanent closure (or its trial) in which their own capital is at risk, the exchange will, clearly, be more benefited to comply and give the legal data of their clients. Is because of this that this is commonly referred as “centralized exchanges”\footnote{Although the term \textit{centralized} is not properly applied in this concept.}.

After exposing this paradigm about the market incentives, why there are “accepted” cryptocurrencies and there are other “criminalized”? In the case of Bitcoin, this has already been adopted as legal tender in El Salvador and Central African Republic. Monero, to cite a counter-example, it's being associated with cyber-attacks (hacking, ransomwares, etcetera) or surrounded by some kind of controversy about any illegal activity. Just by citing Cointelegraph:

\begin{quotation}
Privacy coins have been surging lately as it appears that family funds and individuals investors are increasingly holding XMR as a hedge amid recent market turmoil. The topic of privacy coins has been controversial among the crypto community. Some point to their ability to ensure greater anonymity during transactions, while others raise concerns about using XMR to shield illicit transactions and its alleged embracement by extremist groups. Last year, Kraken delisted XMR for its U.K. customers, citing regulatory pressure.  \cite{cointelegraph}
\end{quotation}

Beyond the obvious clarification about a good not having any kind of morality at all\footnote{In this case, Monero is not “immoral” because it's used by criminal. Immorality resides in the individual's action that threats the life and/or the private property of others. The asset \textit{per se} has no intrinsic morality.}, Monero marginalization might be because of the actual impossibility of tracking the transactions in the blockchain; and given its internal protocols of safety to avoid any kind of attack is what makes the States, unlike Bitcoin, prohibits it (it cannot be controlled whether directly or indirectly).

In order to respond the central issue of this section: legality does not implies morality. In the academic discussions about \textit{ethics}\footnote{Precisely, about the categorical imperative exposed by Immanuel Kant.} and legality, it's common to cite the example of a random individual who helps to refugee a jewish in the National-Socialist Germany of the 1940s. Helping a person, despite its religion, is a morally “good” act, but in this case it goes against the German law and the duty of the citizen. However, handing over the jewish to the German authorities it's a morally “bad” act (the person will be tortured, disappeared, shot, etcetera), although this implies the respect over laws and the duty of its role as a citizen. Despite the trivial example, this serves as a counter-example about the law compliance issue and the morally good acts that are incompatible.

In conclusion: the user must not be concern about if its actions are \textit{legal} or \textit{illegal}, but if its actions are morally \textit{good} or \textit{bad}. This distinction it can be done as long as it doesn't affect the life and/or the private property of a third party. Without going further, the State functions through violence (a morally \textit{bad} act) through restrictions on the individual freedom, being this in the economic plane (taxes, regulations, depreciation of the national currency) as well as the freedom itself (understand this as the arbitrarily restriction imposed during the COVID pandemic); even expressly threatening the lives of people (for example, the Russian invasion in Ukraine).

\section{Entrepreneurship}
Starting a business is a complex process. This starts from the initial idea of what one could offer to the market, through the image of the new business, even its marketing. Regarding to the manufacturing process or service, one must consider the raw materials and the available supplies for its execution, \textit{id est}, the initial capital for starting a business (including the \textit{material} capital and \textit{intellectual} capital). And regarding to its application, one must think, with such initial capital, in the manufacturing processes or services to create a \textit{standardization} of the work processes.

Within the \textit{legal} framework, it's difficult to establish “equal” parameter for each persons who lives around the world, because the legal requirements are different for each country. Some of them are more “flexible” to start a business; and some others require a long and tedious bureaucratic process\footnote{From the decision of starting the legal process until, finally, trade with the State permission, one can spend half day of paperwork in New Zealand \cite[p.~4]{db:newzealand}, or 230 days in Venezuela \cite[p.~4]{db:venezuela}.}, in which one has to spend a lot of money for its validation (even, being involved in acts of corruption). Given the agorist \textit{ethos}, generally speaking in the cryptocurrencies communities in which it's assumed this asset as a medium of exchange, business are created to grow \textit{parallel} markets\footnote{This means, an \textit{independent} market from the \textit{fiat}-based economy, in which cryptocurrencies are used as money. It's used the word “parallel” and not “illegal” because there's a rapid negative connotation towards the word “illegal”; but, as mentioned in the previous section, legality is not a valid parameter to establish the morality of human action.}, and overpass all of these state controls.

In this sense, trading to offer goods and services in the \textit{legal} or \textit{parallel} market it's “the same thing”. Obviously, for the manufacture of certain specific goods, there will be greater state controls and restrictions towards private actors, depending on the magnitude of the invested capital to produce it (it won't be the same cooking bread and sell it in the parallel market than fabricate an open-source car and sell it). Beyond these differences, there is no requirement of any process or special validation to start a business and sell in the parallel market: in fact, conceptually speaking, there will \textit{always} be a higher degree of freedom in this market than the \textit{legal} one.

\section{Open Source Brands}
\textit{Open Source Brands} would be, in simple terms, an already created model business: both the “image” of the brand and the entire standardized process of manufacturing/service is already settled in digital documents. This kind of business model is not, by any mean, a necessary requisite to create such parallel market. In this document it is presented \textit{another} way to run a business in order to encourage crypto markets to detach from the exchange rate of the cryptocurrency in \textit{fiat}\footnote{About this, is referred as the act of buy cryptocurrencies “when the price is low” and sell it “when the price is high”. In this exchange, it's used cryptocurrencies as a financial asset to earn more money; and, despite the fact that it could contribute to a higher liquidity in the market, this surely doesn't contributes to the creation of a parallel market at all.}. Basically, it's about to offer a \textit{business model}, in an “open source way”, in exchange for a fee for each sale (this is analogous, in the actual economy, to create a patent of a good, so that a company can take such patent and fabricating on its own).

Why, then, speak of an \textit{open source brand}? Given the experience of the development of \textit{open source} software, the work methodology behind this process (public scrutiny and renewal of processes) could be, also, applied to projects \textit{outside} the digital world. Just as it has been taken Bitcoin as a standard model of cryptocurrency, and it has been made different modification of the code to launch new cryptocurrencies, the same could happen with open source brands. This means, no only one could offer brands of the same category, competing with themselves; but it can also rise new brands that offers \textit{very} similar products between themselves, competing with the price and fees offered in the market (this last point will be developed in the next paragraphs).

It's pretended that, in the long run, there could be a situation in which people around the world, that wants to start \textit{immediately} a business, can choose different brands within a “brands catalog”. And so, brands that please them, or it could be economically more viable in the location they are, can be taken and used under the standardized work processes that the same brand stipulates, to sell it in the parallel local market. This idea would suppose, immediately, a reject due to the economic incentives on the creator of the brand to be commercialized; but this will also be exposed in the next paragraphs.

It's also expected that the manufacturing and sales of these products can be valued in different cryptocurrencies (and not just in a \textit{single} cryptocurrency, although this will depend on the subjective preferences of the seller); leading to a greater liquidity in the market. And this will provoke different market opportunities for new brands and competitors to emerge, and the prices of the offered goods and services to be regulated automatically in the market.

\subsection{Advantages}
What advantages an open source brand could have in comparison of a common business? Technically speaking, an open source brand \textit{is} a business; and this will not suppose a change in the paradigm that leaves “obsolete” a common business model. Just because it doesn't have much difference with the conception of an open source software, it could be mentioned some advantages when at the moment of creating an open source brand:

\begin{enumerate}
\item Transparency in the work processes.
\item Public scrutiny of the brand.
\item Skipping State controls.
\end{enumerate}

\subsubsection{Transparency in the work processes}
Just like in the digital world, where programs are under an open-source license in which any person can have access to such code and make a scrutiny of its operations, the same it is proposed with business and its work processes. In what thing can this feature differentiate with a traditional business? The difference would be the \textit{transparency} of the work processes (because these are publicly described).

Given that the word \textit{transparency}, in the crypto environment, is referred to the visualization and tracking of the user's transactions, here is referred to the work processes. It has nothing to do with publicly show the user's account whatsoever, it only specifies about \textit{how} to work with the raw materials.

This \textit{transparency} of the work processes isn't something exclusively in the digital world, but rather also in the real world. An everyday example is in the gastronomic world, where in internet and in diaries, magazines and books are established the raw materials and the work processes to cook a specific meal. When someone checks a recipe from the book \textit{Gino's Italian Express} by the Italian cooker Gino D'Acampo, it's being working with the raw material and the cook process that has been stipulated by Gino D'Acampo himself. Instead of personally cooking the meal from the user who desires it, he leaves the entire work process in a book for the user to buy it and cook the recipe (and obtaining a fee from the book sale).

The business model of the open source brand, conceptually speaking, refers to the same thing\footnote{Another example of \textit{transparency} is when parents, when they take their son or daughter to a private school, they desire is to see the curriculum taught in that school. Depending on their subjective preferences, and the comparison with other private schools, parents could take a decision about it and they'll choose one institution over another (when the economic factor is discarded). The standardization of the teaching process is “materialized” in the curriculum: the process can be done partially or totally; but this is a guide for the parents to know \textit{a priori} what kind of content it will be taught. In the opposite case, the content will be revealed \textit{a posteriori}: they'll only know the curriculum of the day, after being taught to its son or daughter. Therefore, in this case, the \textit{transparency} is both beneficial for school and parents. The institution will make sure a monthly income in the long term, and the parents will make sure of paying for a good long-term service. Besides, there's an extra advantage in this scheme that allows competition of the curriculum among different schools.}. By detailing the work processes to elaborate certain goods or services, a third party cannot only have a higher degree of security by knowing \textit{how} it's been elaborated, but also this third party will be able to replicate this to offer the same product.

\subsubsection{Public scrutiny of the brand}
The same thing that happens with the \textit{open} software, in this case it will also be able to make a revision on the entire work process and raw materials (it's worth to mention that public scrutiny is only possible when there is \textit{transparency}). This revision could derive in cases where interested people can correct certain errors overlooked by the inventor. In the case of not correcting it, given the \textit{open source} nature of the brand, this could be an opportunity for \textit{alternatives} brands to emerge with a similar process\footnote{This is what happened, for example, with Bitcoin and the cryptocurrencies derived from the same code.}.

Public scrutiny, applied to brands that offers non-digital goods, can serve to replicate the work processes and publicly object if it's feasible or not (and if the produced good is \textit{legitimate} or not). Also, it could also serve in case where the inventor has designed and created a good, but he hasn't the necessary capital to manufacturing it. The most simple example, in order to understand this idea, could be about the 3D printing market, when, maybe, a person has designed a manual machine, but he can't buy a 3D printer due to its high cost (and its lower purchasing power to acquire it). Given the fact that is an open source brand of 3D printings, being a complex machine, it could be left to a public scrutiny for the model to be analyzed, replicated and made the necessary revisions in the case where certain flaws may appear\footnote{This is similar to having a company in which the new designs are sent to the R\&D Department.} (and that people know whether this design works or not). Once the design is validated, the brand could be used to fabricate such machine.

This concept of the public scrutiny, also, can be of a great help when one must fabricate a good which requires a huge initial capital but, for example, in the location where the inventor lives, given a greater degree of authoritarianism and State control, it's not possible for him or her to fabricate it (both legally and economically). Here, the quality control of the design could be done \textit{outside} the State surveillance (this means, in “freer” locations) to let the inventor know if its project is free of flaws and if it is feasible its invention.

For example, a person located in Equatorial Guinea, who is interested in creating an open source brand of vehicles (with a specific logo and a standardized fabrication process), digitally designs a car. Her problem lies in not been able to materialize it because of her lack of resources and a huge amount of legal obstacles in order to fabricate it and testing it. By publishing her design and stipulating it as an open source design, an Argentinian could fabricate it and check it for a quality control. A Mexican could check this quality control and arriving to the conclusion that this data is inconsistent, being able to be done a minimal change in the design in order to be more efficient. Given the public scrutiny of this design to improve it, the Ecuatoguinean could make the final design in a new version; and she would publish it for its manufacturing (in exchange, of course, of a fee for each sale of the fabricated car).

The previous example could be easily rejected by its “idealism” by pretending to create car and believe in the idea of circulating freely in the cities, away from the police and transit control. Although it's true in the current circumstances, the issue here is based on the scrutiny and quality control of a good that requires a great capital, in which the creator can't afford with its own means. However, the example of the open source car's design it's inspired in the real case of \textit{Hyperloop}, a proposal designed by Elon Musk of a high speed rail system, in which the train travels in a vacuum tube, published as an open source code in October 9, 2013 \cite{hyperloop}.

And so, what would be the benefit of a public scrutiny? Continuing with the example, if there's interest in acquiring a car valued in cryptocurrency, the same persons interested will want a car with the highest expectations (clearly, for their own security and specially for their own interest), the benefits that they desire, and the acquisition of an asset beyond any king of State control\footnote{Because of the quality control and homologation is being done by the State, given the incentives it haves, it may incur in arbitrarily expensive and extensive bureaucratic paperwork, arbitrary legal and/or fiscal impediments in production, as any kind of corruption act in the process (bribery, tax investigation, arbitrary fines, union persecution, etc.).} (with the advantage that supplies the fabrication and circulation of goods in the parallel market).

In summary: public scrutiny will allow a more rigorous quality control for everyone interested in acquiring such good, instead of trusting or relying on a State controlled company, or even a State company, in a specific country.

\subsubsection{Skipping State controls}
This point will be what would differentiate an open source brand from a common business. Before continuing, State control will be impossible to \textit{completely} avoid it in the business, but it can be reduced to its minimal expression; and as the authoritarianism is greater, the harder will be to avoid this State control\footnote{Although this declaration could be subject of debate, given the possible justification that greater the State is, more inefficient the State will be to control all of the human activities.}. Having said this, an entrepreneur could doubt about the open source brand; and being the entrepreneur itself who produces goods and trade it on a local level. Nevertheless, the problem, here, is the trading in a bigger scale.

A business in the parallel market could be restricted by the geographic location in which is located. The businessman, therefore, will be able to trade his/her product locally but with the difficulty of expanding and avoiding the State control. If, for example, a person decides to buy X good, living in a foreign country, an alternative could be sending such good by any means of transport (for example, by air or by sea). The big problem the seller will encounter will be the following:

\begin{itemize}
\item State bureaucracy and necessary permits.
\item Customs corruption.
\item Payment of shipment in \textit{fiat} money.
\end{itemize}

The first point refers to the entire homologation process and the approval of such good to certificate its production with the national and international norms. This means, apart from doing paperwork, customs will value, classify and verify such good to approve its shipment to another country; or prohibits if this good is considered “dangerous” or “illegal”.

The second point, as mentions before, deals with the corruption acts the custom may take: extensive delays to deliver it; bribes on the client to give the stored package; misplaced permits that prohibits its withdraw; payments of arbitrary fines or fees on the service; among other things.

The third point sums up all of above, and refers that the shipment, being controlled by the State, it's a service paid in \textit{fiat} money; and this is opposed with the idea of creating \textit{parallel} markets, with the end of trading with cryptocurrencies and abandon \textit{fiat} money. Here, the problem lies that the valuation of an asset in cryptocurrency will loose its importance, because all of the intermediate processes between the seller and the client will be valued in other currency. This implies that, given the situation of rising the costs in another currency, it will end up being more practical to sell it in \textit{fiat}, rather than doing the exchange process of the cryptocurrency.

The open source brand will allow to replicate the production of a good in a single location to another, skipping all of the shipment's State control, but also allowing the creator of the brand to earn money with it. In this way, the geographic restriction will be solved; and despite the fact that the creator won't be able to produce and ship it to a specific location, he or she will be able to earn money through fees in order to “validate” such creation (but the economic incentive issue will be explained later).

\subsection{Disadvantages}
It must be pointed out two \textit{fundamental} problems about the open source brand presented in this article. To begin with, this business model differs from an open source project based on donations, this means: a developer or a group of developers that receives money voluntarily from the members of the community because of their word; and such donation doesn't interfere with its use. On the other hand, also differs in the business model based on CCS (Community Crowdfunded System), in which an idea is proposed to be funded, or not, by the community. Here, in the open source brand, a brand is proposed and developed; and it's handed over to the community in exchange of a fee for every sale of each good. Therefore, it can be enumerated two \textit{big} problem when creating a business with this characteristics:

\begin{enumerate}
\item Product counterfeit.
\item Economic incentive.
\end{enumerate}

Given that the open source brand establishes the procedures to fabricate a product, with the necessary raw materials and supplies, one person could just simply search such procedure, create the stipulated product and not paying any fee to the creator. This scenario would “force” the creator to \textit{trust} in the producer's morale to pay such fee for using the brand. In this article, it will be presented its respective solutions that will serve as a starting point to run the business. The experience, in the end, will determine the best possible solutions to carry on with this business model.

\subsubsection{Counterfeit}
It's necessary to clarify that there's a \textit{substantial} difference between counterfeit and replicas. Offering a counterfeit good and offering a replica to the market it's not the same thing.

Respecting replicas, these cases are common in real life. For example, when creating the recipe of Baileys Irish Cream, there's a huge amount of different recipes on Internet that replicates such drink. All of them, with the necessary raw materials (cream, condensed milk, chocolate, vanilla and Irish whiskey) in different proportions. However, in these cases it's being achieved a very similar product (to the point of being marketed as a Baileys-like drink), but none of them belongs to the original brand that created such recipe. Here is where one could consider the idea of “added value” at the moment of buying a  drink certified by the brand itself, and buying a replica of it.

Citing a bigger case, in the effect pedals market, there are a hundred replicas of guitar effect pedals that are considered “classic”\footnote{Some of these pedals can be: Dunlop \textit{FuzzFace}; Vox \textit{Wah}; Univox \textit{Uni-Vibe}; MXR \textit{Dyna Comp}; MXR \textit{Phase 90}; Boss \textit{DS-1}; Ibanez \textit{TubeScreamer 808}; ProCo \textit{Rat}; etc.}. The company by the name Behringer, for example, it's known for having a brand of cheaper pedal effects that replicates “important” and more expensive pedals. Josh Scott, the manufacturer of JHS Pedals, in his Youtube channel made a comparison between these pedals and their Behringer replicas. Next, a comparison table of some of the pedals that Josh considers as an “exact replica”, with their respective price market:

\begin{center}
\begin{tabular}{|c|l|c|l|r|}
\hline 
\textbf{Original} & \textbf{Price} & \textbf{Behringer} & \textbf{Price} & \textbf{Diff.} \\ 
\hline 
Ibanez CP-9 & USD 170 & CL-9 & USD 25 & x6,80 \\ 
\hline 
Boss TR2 & USD 114 & UT300 & USD 25 & x4,56 \\ 
\hline 
Boss GE7 & USD 137 & EQ700 & USD 25 & x5,48 \\ 
\hline 
Boss VB2 & USD 254 & UV300 & USD 25 & x10,16 \\ 
\hline 
Boss CH1 & USD 137 & UC200 & USD 25 & x5,48 \\ 
\hline 
Ibanez TS808 & USD 165 & TO800 & USD 25 & x6,60 \\ 
\hline 
Boss FZ2 & USD 370 & SF300 & USD 25 & x14,80 \\ 
\hline 
\end{tabular}
\end{center}

\begin{footnotesize}
\begin{flushright}
Extract of the video \textit{What's the Deal With Behringer?} \cite{pedals}
\end{flushright}
\end{footnotesize}

Despite the replica of the original circuits, achieving an \textit{identical} sound in many cases, pedals are sold with the Behringer name. The \textit{radical} difference respecting counterfeit would be that Behringer fabricating such replicas, selling it as being original\footnote{This means that Behringer would manufacture its UV300 pedal; but instead of using its own case, use the Boos VB2, replicate the same color and typography, and sell it as an original.}. In this case, it would be deceiving the client by selling a replica as it would be an original. Different the matter would be if, despite of being a replica, the client knows that he or she is buying an “identical copy” of the original. Here, therefore, the person will know that he is buying a replica (a copy), not an \textit{original}.

Given the fact that the processes in the open source brand are already transparent (this means, “it's on plain sight of everyone”), maybe there would be no incentives to create counterfeit of such product\footnote{Assumption made on the mentioned case of effect pedals.}. In any case, one could offer a similar good. Nevertheless, it will be assumed that a counterfeit market will be inevitable\footnote{See the business model case of privative software; and the \textit{crack} of its code to download and using it without paying any cent for it.}.

In summary, the first and most obvious disadvantage of this business model could be \textit{counterfeit}. Going further, the malicious actor could make, for example, a marketing campaign on a specific brand as the one who provides a “dangerous”, “deceitful”, or “defective” product, (thus, associating the brand with some of these characteristics to “remove” it from the market). Nonetheless, given that the brand and the manufacturing process is being held in the public scrutiny (being, therefore, easily accessible), this situation could be easily objected: it would only be suffice to look up the raw materials used and the manufacturing process to determine whether the final product is genuine or not.

Although the next can happen outside of the economic sphere (and exceeds the proposal of this article), it could exist the situation in which some malicious actors, in order to “eliminate” a brand, insults the creator of the brand to discourage its production (and the earning that the creator receives through fees). It could happen that, for example, John doesn't like Anthony, the creator of a brand. His resentment is such that decides to boycott the brand by falsely denounce in social media that Anthony is a \textit{nazi}; demanding the community to cease the production of the brand (in order to stop financing a person with “a nefarious ideology”) and, also, promoting censorship of any new brand that the creator invents\footnote{This situation could serve as an opportunity to create a \textit{Private Justice}, with the goal of solving disputes and grievances against private property (extending it to the notion of \textit{integrity} and \textit{honor}).}.

A case in which it does competes to this article is whenever a malicious actor, with the objective of eliminating a particular brand, instead of offering a similar and competitive product, manufactures such good but with different material (or with a different manufacturing process) with the goal of purposely creating a “defective” product. And so, he will be able to insert a counterfeit product in the market for the clients to buy it and associate the brand to products of “lower quality”, “dangerous”, “toxic”, “not trustful”, etc.

How can the client be sure that the product that wish to buy is “authentic” and not a counterfeit? In principle, it could be considered the creation of a company that offers certifications of quality control. This means: an \textit{autonomous} company that verifies both manufacturing process and the final product; and grants a quality certification to both the producer and the product. On the other hand, it could be the same creator who supervise the manufacturing process and the final product to be the same person who grants the quality certificate.

In these two paradigms, it would appear a new problem (which it's nothing “new” at all): one must appeal to a \textit{trusted third party} to determine the product authenticity. Here is where appears, once more, the \textit{blockchain} technology and the theoretical development of Satoshi Nakamoto in his article \textit{Bitcoin: A Peer-to-Peer Electronic Cash System}:

\begin{quotation}
What is needed is an electronic payment system based on cryptographic proof instead of trust, allowing any two willing parties to transact directly with each other without the need for a trusted third party. Transactions that are computationally impractical to reverse would protect sellers from fraud, and routine escrow mechanisms could easily be implemented to protect buyers. \cite[p.~1]{bitcoin}
\end{quotation}

Therefore, it could be used a \textit{blockchain} (whether is a new blockchain or the same cryptocurrency's blockchain) to employ an authenticity mechanism of products, and settle the products certifications. This certification, to give an example, could be the proof of payment with the company or the creator who provides such certification of the final product. Therefore, when the client would like to verify by its own means if the new product that he'll buy is “authentic”, he could be do it through a QR code, in which the certification of the product, and the proof of payment of the producer to the company could be settled in a blockchain\footnote{Again, this is a simple example that must be subject to debate, as a simple way of offering an authenticity verification to the user. Just like receiving or sending cryptocurrency has evolved from copying an address, in plain text, to be shown such address in QR code for the user to do, simply, the transaction, here something similar could be created.

In the section \textit{Verification} it will include a possible form of certification (with its advantages and disadvantages); but, again, this wouldn't imply discarding any new method or applications to verify the quality of the product, given the same interaction between the economic agents that are interested in buying a “trusted” product.}.

\subsubsection{Economic incentive}
This point could be the most controversial of all and the cornerstone of this business model. So far, it has given for granted that those who wants to manufacture the products of the open source brand will be willing to voluntarily pay a fee to the creator for each sale. But this implies that the producer could sell the product and \textit{choose} not to pay to the creator (and being an open source brand, there are no privation whatsoever for the producer to do this). This generates a situation in which this isn't favorable for the creator\footnote{Unless, of course, that they offer an open source brand and explicitly says that he doesn't want to have any earning at all.}.

By not being able to ensure a constant income through time, this business model seems to not providing any incentive economic at all to create a brand; and, therefore, somebody who wants to manufacture such good, and sell it in order for both parties to earn money. The only situation in which the creator could earn money would be in the case where he could ask for donations to the rest of the community (as mentioned above).

To solve this problem from a \textit{purely} economic point of view, it could be referenced the business model the mining pools have\footnote{In fact, some of the pools are open source brands because, literally, its functioning code is in public scrutiny.}. For a small percentage fee, the pool offers servers for different people around the world to gather and mine cryptocurrencies; and, thus, rise the chances of earning the mining reward (and this would be distributed for each participant, depending on the reward system offered by the same pool).

Now, what parameters could be taken in account for choosing one over another if, ultimately, all pools offers the same service (mining blocks)? In principle, the fee between the same pools\footnote{There are, of course, \textit{exceptions} to this, like P2Pool in Monero.}. If all of them offers the possibility of mining the same cryptocurrency, regarding to its economic incentive, it will be the percentage fee when a user wants to withdraw his money. Next up, a list of some pools, with their respective percentage fee and the lower limit to withdraw XMR:

\begin{center}
\begin{tabular}{|c|c|r|}
\hline 
\textbf{Pool} & \textbf{Fee} & \textbf{Min. Payment} \\ 
\hline 
MineXMR & 1,10 \% & 4,00 mXMR \\ 
\hline 
Nanopool & 1,00 \% & 110,00 mXMR \\ 
\hline 
XMRPool & 0,90 \% & 70,00 mXMR \\ 
\hline 
Hashvault & 0,90 \% & 1,00 mXMR \\ 
\hline 
SupportXMR & 0,60 \% & 100,00 mXMR \\ 
\hline 
MoneroOcean & — & 3,00 mXMR \\ 
\hline 
P2Pool & — & 0,30 mXMR \\ 
\hline 
\end{tabular} 
\end{center}

Notice that MoneroOcean and P2Pool doesn't charge users. Yet, they differentiate each other in which MoneroOcean allows a minimal payment of 3 mXMR, while P2Pool allows it at 0,3 mXMR.

Apart from the competition between these pools, this could be taken as an example to the business model. The seller, when he offers his product to the market, could give a QR code for the client to scan it and pay it, with the difference that within the same QR code it has the order to pay both the seller and the creator. For example, given a liquor brand, the seller could label the bottle with a single-payment QR code for the client to transfer, with his corresponding \textit{app}, the money to two accounts in a single time. If the liquor brand asks for a 1\% fee for each sale, then accounting speaking, it would be shown in the following way\footnote{It's worth to clarify that here it doesn't take account of the transaction fee of the cryptocurrency.}:

\begin{center}
\begin{tabular}{|l|r|r|r|}
\hline 
\textbf{Agent} & \textbf{Debit} & \textbf{Credit} & \textbf{Percent} \\ 
\hline 
Buyer & — & 25,000 mXMR & 100,00 \% \\ 
\hline 
\hline
Seller & 24,750 mXMR & — & 99,00 \% \\
\hline 
Creator & 0,250 mXMR & — & 1 \% \\ 
\hline 
\end{tabular} 
\end{center}

If the seller would offer the QR code to the buyer to make the transaction, it could be like this:

\begin{center}
\includegraphics[width=0.5\textwidth]{media/qr-code-en.pdf}
\end{center}

With a single image, the QR code will show:

\begin{itemize}
\item The total amount.
\item The seller's address, with its corresponding amount.
\item The creator's address, with its corresponding amount.
\item The detail of the purchase.
\end{itemize}

What would happen if the seller would, maliciously, deceive the buyer by giving two different addresses but all of them correspond to the same seller? To reduce this trust degree and avoid this act, both the creator and the seller will must publicly offer their addresses in a \textit{.asc} file, signed with their respective private GPG keys, and use a protocol to verify that the included addresses corresponds with the involved parties in the QR code. In this way, the buyer, by doing a visual comparison or through a simple \textit{app} (or feature), he could verify that both directions corresponds with the seller and creator.

Following the example above, now with real addresses and with orders created through the option \textit{Payment Request} in the Monero GUI program (although with a modification in its syntax), the QR could look like this:

\begin{center}
\includegraphics[width=0.75\textwidth]{media/qr-code2-en.pdf}
\end{center}

This code breaks down as following:

\begin{itemize}
\item \textit{Seller} data:
	\begin{itemize}
	\item \textbf{Monero address:} 8BB dgnA ckd2 DWfX qHcc 2466 v19d VeWv hgAt jSm1 hZGP xe6C BkPg Dxqa BLbr D9NL K8re WLFf oUtd pg18 rzkT DC2i BJMe rKDv
	\item \textbf{Quantity:} 0.0247500000000
	\item \textbf{Name:} Seller
	\item \textbf{Description:} Liquor bottle
	\end{itemize}
\item \textit{Creator} data:
	\begin{itemize}
	\item \textbf{Monero address:} 887 UEV7 9zZT KeKb tGjT 2GUe 6P6H aFLn DF52 KibH PL2C yZUp isgA 2EyN anJh LXRf oJW6 FNhX Q7sd h9SE K8YX u7ZX 8JKL YADh
	\item \textbf{Quantity:} 0.000250000000
	\item \textbf{Name:} Inventor
	\item \textbf{Description:} Li'monero
	\end{itemize}
\end{itemize}

And so, the client will have the next parameters:

\begin{itemize}
\item The total amount of both transaction adds up to 0.025 XMR (or 25 mXMR), amount that is equal to the sale price of the product.
\item Those 25 mXMR are broken down in 24.75 mXMR (99\%) to the seller and 0.25 mXMR (1\%) for the creator, both prices matching with the fee of the creator according to the total price.
\item Through a verification of the public GPG keys, the customer will check the validity of the addresses (and this will be only checked if both recipients has given publicly their address, signed with their own private keys).
\item The customer will be able to see the name of the address and the description at which corresponds the amount of the transaction.
\end{itemize}

In this way, the buyer will be able to do it rapidly (being through a computer or any mobile device), without the necessity of making too many steps for its validation. Also, this could be done by the creator itself.

In summary, this examples contains the following advantages:

\begin{itemize}
\item The buyer will have the easiness of paying for the product, without the necessity of having to pay both seller and creator individually: he will do it in a single payment.
\item The seller will have an economic incentive, because he will have a great percentage for every sale (in this example, a 99\%).
\item The creator will be able to an income, in a \textit{passive} way, with each sale, by offering his idea to the market. This will encourage competition in the market for more people to be able to manufacture its product; and obtaining, by so, greater passive income.
\item All of the economic agents will be able to verify, through the QR code, that the corresponding addresses with their corresponding fee are \textit{reliable} according to the GPG keys. Otherwise, some of them could claim to the other party the commitment of the agreement; or, simply, the product won't be sold because it doesn't accomplish the requirements.
\end{itemize}

\section{Verification}
The verification of non-tangibles assets, given the \textit{digital} nature of the cryptocurrencies, is very simple.

For each digital asset created, the producer could do a hash verification of the file\footnote{Commonly, it's used SHA-256; but it can be done with any other hash function, such as BLAKE2, BLAKE3; or the new family of SHA algorithm, being SHA3-256 or SHA3-512. Even, it could offer several hashes to do a “double check”.}, and offer the easiness to the user to check the file's integrity. To give an example, the hash of the program XMRig, in the version 6.17.0 for Linux x64, in the .tar.gz file is:

\begin{center}
\texttt{75ce 5d4d 52c4 6a7c 8c60 4e1d e354 9cba 9dc4 b074 05d6 598e 12b6 f21f 5024 7739}
\end{center}

Using a program such as GtkHash\footnote{Programs that are written to do a hash verification (instead of writing the order \texttt{sha256sum} in a Linux terminal, for example) could be considered like some sort of “trusted third-party”. The trust degree will depend if the program is open source or not, in order to do a public scrutiny of the code.}, the downloaded file could be open in the program, and do the hash verification. If this is correct, the program will mark as a valid verification through the SHA-256 algorithm. Otherwise, it won't mark any box, and the user will have to check if the file has been corrupted when downloading, or the hash has been written with errors.

This practice, therefore, could be easily done, and it won't depend on any company or third party for the digital file verification: even, this document distributed in GitHub is shown its hash sum for the user to be sure of the authenticity of this article.

And so, how can expand this idea in the purchase of tangible assets? The use of the SHA-256 in files exists to check if the hash sum matches with what the company or person publicly shows. Nevertheless, this could be subject of hacking (this means, a hacker could intercept the website where this hash is being stored, or the communication between the client and the server) the security must be even stronger. Given the proposal that different parties involved in a transaction must share the address of their wallets, any malicious actor could intercept this communication and “deceive” the client. And because of not having a trusted third party, if the client sends their funds to an unknown account, he will irreversibly loose his funds. In summary: a situation in which the involved parties provides, in plain text\footnote{This means, without any type of encryption.}, is not a condition enough to \textit{make sure} that the address is legit. Here is where the GPG protocol appears.

Given the fact that there's no third party to settle up the transactions among people, every party involved could publish their addresses through the GPG protocol. In that way, each person could publicly share their public key; and the people who desires to make the transaction they will be sure that the address corresponds with the person or company involved. This, maybe, could represent a “disadvantage” because the clients, when they buy for the first time, they'll have to take their time to search the GPG signature of the other party. However, this will represent another security step; but it won't represent a huge setback. In fact, once a public key is registered in the digital device, this step won't be necessary to do it again.

Of course, this procedure must be done \textit{before} the payment. Through the app to transfer crypto, the program must have, first, the public GPG keys of all of the parties involved (seller, creator, etc.). Then, the program must verify that the \texttt{.asc} file, containing the signed address of one of the parties, matches with the provided public key. If all of this is correct, the payment could be enabled.

Besides, the GPG protocol could be used to specify the respective fee percentage of each of the parties. If though a simple math calculation, for example, the fee for the creator is 0,1\% when the brand publicly establishes a 1\%, the operation will be rejected. The client, therefore, would act as a mediator by looking the terms between the seller and the brand, and rejecting the purchase (situation in which the client could suspect of the authenticity of the product and, thus, the seller). And so, a degree of trust is eliminated between the seller and the creator by being the customer a \textit{passive} actor of mediation.

In the end of this article it will be annexed a \textit{basic} flow chart of this proposal; and a folder with both the public key in a \texttt{.asc} file and a \texttt{.txt} file with the hash sum of this document, with the corresponding signature, will also be added in the GitHub account for the user to verify the authenticity of this document.

\subsection{Procedure}
For those who doesn't know how to check a GPG key, it will be necessary to have a program that could decrypt and check the GPG signatures. The most known program is \textit{Kleopatra} (although there are other alternatives such as \textit{GPA} in Linux), and it's a GPG key manager. Therefore, the user, if he wants to check the authenticity of a message or file, like it was said, he will must have the public key of one of the parties. And so, the signed text file, provided by one of the parties, could be easily verified. One just simply must follow these steps:

\begin{itemize}
\item Importing the public key to the corresponding program.
\item Selecting the file text that contains the signed message.
\item Verify the matching signature with the public key.
\end{itemize}

If this produces an error (this means, the program doesn't recognize the signature) the text file will be adulterated.

\section{Ramifications}
The incentive to create more business to offer goods and services to the parallel market, by demanding raw materials and/or other services, will provoke a local market opportunity for the rise of \textit{others} business to supply such demand. For example, in the case of a liquor rand, the demand for raw materials (such as glass bottles, white drinks, fruit and sugar) will allow the opportunity for glass-bottle, white-drink and agricultural producers to rise in the market. At the same time, all of them will demand supplies to create its products: the glass-bottle producers will need glass (their raw material), but also industrial furnaces and fuel; the white-drink producers will also need their raw material, and also the supplies for fermentation and storage; the agricultural producers will need water, agrochemicals (if their production method is not “agrochemical-free”), fertilizers (chemical or natural), and agricultural machines, etc.

It is estimated that, in the beginning, there will be very few business (and in a less quantity, open source brands) that they'll have to deal with the “volatile” exchange rate of cryptocurrencies\footnote{Again, because of the extended notion of seeing cryptocurrency as a financial asset to do \textit{forex trading}, instead of being used as \textit{P2P cash}.}, the low-margin earnings by sales, and the discomfort of dealing with two types of currencies to obtain the raw materials and supplies. Given this situation, the answer will be in the decision of each economic agent in taking the risk of starting a business or not.

Nonetheless, it's worth to mention that these brands could be better applied in countries with great economic instability, generally speaking, because of the low flexibility of foreign currencies and a rapid depreciation of the local currency. Cryptocurrencies, therefore, will be a “less” volatile mean of exchange, comparing it with the local fiat currency; because the long-term tendency of the State currency tends to depreciate, while the long-term tendency of cryptocurrencies generally tends to appreciate\footnote{In the year 2009, Bitcoin was worth absolutely nothing; but today, in 2022, have the highest price in the entire market. The same happens with other cryptocurrencies in their beginning, respecting today.}.

Going back to the example of the open source liquor brand, yes: in the beginning it could be a lower margin of earnings (or even none due to the purchase of glass bottles, their corresponding caps,  and the printing of the labels for the brand. But in this case, a “twist” could be done: in the beginning, the glass bottles could be recycled from other empty bottles that people have thrown in the garbage. The original label could be removed for using the bottle, properly sterilized, to be filled, capped and selling it to the public.

The fact of collecting those empty bottles, separating them according to their color, could lead to the creation of glass-bottle factories, in which their raw material is recycled glass bottles thrown in the garbage. Given the fact that glass is reusable, these glass-bottle factories would save their money in the payment of raw material in fiat, to only obtaining it from garbage collection. In summary: due to the demand of glass bottles valued in crypto, a glass-bottle factory could supply this by obtaining the raw material from garbage.

Exploring beyond this situation, it might be the case in a region where there are electric cars, a user who generates his own electricity to charge the battery car, could have the opportunity of transport the raw materials. In this context, the user could value its transport services in crypto, what would cause a competitive advantage for both the liquor producer and the glass-bottle producer, because both may cover a larger sales area.

It could continue with the examples; but as a final comment, suffice it to say that the production of an open source brand (being a specific model business) would also give place for competition between different users: if the prices offered for each of the brands are “too high”, it could emerge other users that produce a similar product, but with a more competitive price. Hence, the idea of \textit{ramifications} of an economical activity in other economic areas which they complement each other.

\section{Liquidity}
\subsection{Local}
Another positive factor of \textit{open source brads} is that would allow a greater circulation of cryptocurrencies. This would imply that, in one hand, there will be suppliers of goods and services (that, in countries where there are less economic freedom, this will be a solution to the exchange problems of the local currency, respecting foreign currencies, and its depreciation); and, on the other hand, there will be customers of those goods and services. Therefore, the economic transactions between different agents will allow that they could have a better accessibility of cryptocurrencies by having a greater volume of circulating money.

This scenario would result in people no longer needing to rely on centralized exchanges (specially, those who are \textit{KYC}) to change fiat money for cryptocurrencies: people will recur to the same economic agents from the new \textit{parallel} economy. And not only that, people will stop treasuring money for the sake of earning more fiat money: it could be used as \textit{P2P cash} (just as being established in the Bitcoin \textit{white paper}). It's worth to mention that, according with the economic theory, this would also increase its value, giving the fact that this would, naturally, rise its \textit{utility} in the market. If this appreciation is transferred to its exchange rate, every unit of cryptocurrency will be more valuable (and the fiat money will be relegated to the tax payment).

This practice could also be transferred to other cryptocurrencies, provoking a crypto competition. In other words, the parallel economy could not be relegated to a single cryptocurrency but many of them. The valuation of different goods will be subject to the economic agents, who will decide which cryptocurrency is the candidate to buy and sell certain goods (or all of them).

\subsection{Worldwide}
In the previous section it was mentioned a \textit{local} scenario; but, being, these brands, \textit{open source}, these will be accessible to anyone in the world. The only barrier\footnote{Assuming that everyone have access to the Internet.}, however, will be the language (and, therefore, the education).

For example, a Venezuelan could create an open source brand of a series of food recipes, with a specific work process. Given that he couldn't access to a bank account, and he's unable to buy cryptocurrency with his local currency because of the extreme depreciation of it in the short term, maybe an Uruguayan could use this brand to sell food; and for each sale, the Venezuelan will receive a small fee, because he gave his address to collect money (although this will vary depending on the business model the Venezuelan decided to offer\footnote{This has to do whether the Venezuelan offered a fee for each sale; or just a single payment by the Uruguayan.}). And so, the Venezuelan will be able to earn money, by offering an open source brand, without relying on a bank account and on an exchange to obtain cryptocurrency. And if this model replicates in the region where the Venezuelan lives, the other inhabitants of that area will be able to get enough money for cryptocurrency to locally circulate.

This will be a greater advantage than just simply trading \textit{locally}. The creators will receive a minimum fee for the sales in other regions or countries; and the seller will be able to obtain a great profit for each sale (that's why the importance of the economic incentive). If one wish to implement a cryptocurrency in a specific country or region, the community could encourage to the user in that area to create an open source brand\footnote{A far more efficient solution than just simply ask for donations.}, and being implemented in other regions and countries (skipping all of the financial barriers and State control).

\newpage

\section{Flow chart}

\begin{center}
\input{media/flow-diag-en.tex}
\end{center}

Valid $ \rightarrow ADDRESS_{GPG} = ADDRESS_{QR} \wedge \%_{GPG} = \%_{QR} $

Invalid $ \rightarrow ADDRESS_{GPG} \neq ADDRESS_{QR} \vee \%_{GPG} \neq \%_{QR} $

\bibliographystyle{apacite}

\bibliography{osbbiblio.bib}

\end{document}
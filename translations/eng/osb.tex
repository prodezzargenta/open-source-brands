\documentclass[12pt,a4paper]{article}
\usepackage[utf8]{inputenc}
\usepackage[english]{babel}
\usepackage[bibnewpage,nosectionbib]{apacite}
\usepackage{longtable}
\usepackage{graphicx}
\usepackage{tikz}
\usetikzlibrary{shapes.geometric,arrows,positioning}
\author{Prodezzargenta}
\title{Cryptocurrencies, and open-source brands}
\date{June 7, 2022}
\begin{document}

\maketitle

\begin{center}
*****[NOT FULLY TRANSLATED YET]*****
\end{center}

\begin{abstract}
Entrepreneurship proposal of an \textit{alternative} business model, in order to create parallel markets by manufacturing goods valued in cryptocurrencies, and to be independent of \textit{fiat} system.
\end{abstract}
	
\tableofcontents

\section{Introduction}
The creation of cryptocurrency has represented, in a higher or lower degree, a rupture in the economy's conception. Since the creation of the State, the economies were progressively administrated by these entities; and through history, it's evidenced how States legitimized the currency monopoly.

According to the article \textit{On the Origins of Money}, Carl Menger says:

\begin{quotation}
To assume that certain commodities, the precious metals in particular, had been exalted into the medium of exchange by general convention or law, in the interest of commonweal, solved the difficulty, and solved it apparently the more easily and naturally inasmuch as the shape of the coins seemed to be a token of state regulation. such in fact is the opinion of Plato, Aristotle, and the roman jurists, closely followed by the medieval writers.  \cite[p. 16]{menger:origins}
\end{quotation}

Giving the fact that money \textit{is and only is} a creation of the State, as a medium of exchange of different goods and services, is absurd. The economic experience about the direct exchange (barter) and its evolution towards indirect exchange (money) is recorded since the times where humanity discovered that metals such as copper, bronze, silver and gold can be treasured and survive the passage of time. Even more, as explained by Ludwig von Mises in his book \textit{The Human Action}:

\begin{quotation}
There were authors who tried to explain the origin of money by decree or covenant. The authority, the state, or a compact between citizens has purposely and consciously established indirect exchange and money. The main deficiency of this doctrine is not to be seen in the assumption that people of an age unfamiliar with indirect exchange and money could design a plan of a new economic order, entirely different from the real conditions of their own age, and could comprehend the importance of such a plan. Neither is it to be seen in the fact that history does not afford a clue for the support of such statements. There are more substantial reasons for rejecting it.

If it is assumed that the conditions of the parties concerned are improved by every step that leads from direct exchange to indirect exchange [...] it is difficult to conceive why one should, in dealing with the origin of indirect exchange, resort in addition to authoritarian decree or an explicit compact between citizens. [...] Under these circumstances there was no need of government interference or of a compact between the citizens. [...] It is certainly more plausible to take for granted that the immediate advantages conferred by indirect exchange were recognized by the acting parties than to assume that the whole image of a society trading by means of money was conceived by a genius and, if we adopt the covenant doctrine, made obvious to the rest of the people by persuasion. \cite[pp. 402-403]{mises:ha}
\end{quotation}

In summary, money is a creation of the same interaction between human being through history, in which they discover by themselves greater advantages than \textit{barter} (or direct exchange) to be able to trade.

The State, in this context, take possession of money administration and, then, the coinage. This situation, added up to the \textit{legitimized} monopoly of violence over a territory, leads up to a situation in which the incentives are not favorable for free trading. This means, on one hand, it's the State the \textit{only} entity, legitimized by itself, to administrate the coinage of money; and, on the other hand, is the entity that enforce violence on a territory.

And so, through the millennia, States has increased their power violently\footnote{At the moment of writing this article, there is an ongoing invasion of Russia on Ukraine, by reasons that has been changed significantly through these months. And this invasion suppose, also, an occupation of the Ukrainian territory to be annexed on Russia.}, both directly (Armed Forces) and indirectly (money). Through the application of taxes\footnote{The word “tax” in Spanish easily reveals the nature of its violent nature: “Impuesto”, that literally means “Imposed”.}, this entity has been able to survive the passage of time; and whoever who don't pay taxes in a timely manner, will be sanctioned in any way.

In the last centuries, when the State began to administrate education over the territory, the theoretical content has been reformed to justify its own existence, financing, among other things, theories that validates its violent acts. And that's how, for example, in countries like Argentina are, still, teaching that macroeconomic imbalances and the depreciation of the local currency (to name some of the actual problems) are purely and exclusively because of the private sector, and not the economic policies created by the State itself. Not only that, but also it's taught this fundamental idea in \textit{every} educational order (from preschool to universities), denying any already checked empirical refutation. This demonstrates that the main \textit{incentives} of the State is to increment its power; and to do this, it must exercise, progressively, more violence against people.

Since the launch of Bitcoin by Satoshi Nakamoto in the year 2009, despite the technical highs and lows that had, the nature of its \textit{open source} code has demonstrated to be a successful implementation for people around the world to contribute, voluntarily, to re-write the code, and improving the original protocol; or by creating new cryptocurrencies based on the same principle. And given the fact that uses cryptographic systems and security protocols, cryptocurrencies (as its name indicates) has been able to establish as candidates to be a medium of exchange of goods and services \textit{outside} the State control. In fact, there are a few \textit{fiat} currencies with the blockchain technology with the end of replacing the actual \textit{fiat} money in a “2.0 currency”.

\subsection{Actual situation}
Despite the creation of many cryptocurrencies, where there's currently registered 717 cryptocurrencies \cite{mps}, there's a \textit{general} tendency to conceive cryptocurrencies as a financial asset to earn more \textit{fiat} money. This implies that these are not considered as a medium of exchange of goods and services \textit{outside} the State control; but rather a mere financial asset to buy it “at low” and sell it “at high”, based upon arbitrary and subjective criteria, or in a sort of technical analysis of its price.

This change of conception can be justified by the multiple ads in the media and social network, based on the navigation history, interaction with other websites, etcetera; \textit{youtubers} financial recommendations (although some of them are not specialized in such discipline); or in the news about how different people “becoming millionaires” or “how they lost all of their money”. However, one cannot ignore the fact that there's not enough economic theory specifically about cryptocurrencies to offer a more detailed lecture of the phenomenon, and also there's an emotional impulse of introducing to the crypto world to earn more local money (or dollars, euros, etcetera) in the fastest way possible.

Beyond the multiple interpretations that it might be written about this (from the user's ignorance until those conspiracy theories, so far, about the State incentive for users to loose all of their money, and emerging as a provider of “financial security” in exchange of a lesser individual freedom), the truth is there's a \textit{misguided} conception about cryptocurrencies. Developers, also, are not extent of this problem. In Monero, for example, the most important development projects are decentralized exchanges (although there is some kind of controversy, yet in discussion, about the internal protocols). This might be justified with the argument of providing even more liquidity in the market; but there are decentralized alternatives already to exchange \textit{fiat} for Monero.

The point in this last statement refers to the central issue of this document: stop perceiving cryptocurrencies as a mere financial asset to acquire it when the exchange rate to \textit{fiat} is “low”, and sell partially or totally the crypto amount acquired when the price is “high” (and, thus, earning more \textit{fiat} money). This way, the human action will tend to see cryptocurrencies as a medium of exchange outside regulation and State control, and produce goods and services under a \textit{parallel} market.

\subsection{Legal issues}
This might be a controversial subject to discuss about the “crypto nature”. Without going into details on this, given the market incentives, the \textit{general} criteria about cryptocurrencies is the following: any crypto denomination being “accepted” by the States implies that these entities can already control indirectly people's purchasing power (whether being partially or totally)

The \textit{legal} exchanges are controlled by States. This means they have to inform about its patrimonial situation and their clients. For example, in the case of Bitcoin, despite the registered transaction in the blockchain of being pseudo-anonymous (\textit{i.e.} there's no registry of the legal data of the user such as name and last name), given certain condition, one can acquire such data. The exchange business model is the following: a financial entity, a third-party, that brings new clients to use their platform; and these entities provides with a private address to offer liquidity and ease of exchange between cryptocurrencies and \textit{fiat} money (in exchange, of course, of a fee by using its services)\footnote{Otherwise, people will have to physically find other persons willing to make such transaction.}. Therefore, this businesses will be in charge of treasure many cryptocurrencies to be able to be rapidly exchanged by its users. However, for users to start trading, it's a necessary requisite for clients to provide their legal data. Binance, the largest international exchange in the crypto world, requires the following data to take away any restriction to its customers for using their services:

\begin{itemize}
\item Country of residence, and nationality.
\item Name and last name, and birth date.
\item Address, city, and postal code.
\item Scanned image of the front and back of the ID, passport or driving licence.
\item Front image of the user for facial recognition.
\item Data verification (several days after the first data entry).
\item Scanned image of any service that proves the place of residence.
\end{itemize}

Taking away the worst-case scenario where the exchange's database is stolen and freely published on the internet, if the exchanges are legal entities that are regulated by the State, what prevents the State to obtain such data, given an arbitrarily situation? If the consequences of not following the State's orders is the permanent closure (or its trial) in which their own capital is at risk, the exchange will, clearly, be more benefited to comply and give the legal data of their clients. Is because of this that this is commonly referred as “centralized exchanges”\footnote{Although the term \textit{centralized} is not properly applied in this concept.}.

After exposing this paradigm about the market incentives, why there are “accepted” cryptocurrencies and there are other “criminalized”? In the case of Bitcoin, this has already been adopted as legal tender in El Salvador and Central African Republic. Monero, to cite a counter-example, it's being associated with cyber-attacks (hacking, ransomwares, etcetera) or surrounded by some kind of controversy about any illegal activity. Just by citing Cointelegraph:

\begin{quotation}
Privacy coins have been surging lately as it appears that family funds and individuals investors are increasingly holding XMR as a hedge amid recent market turmoil. The topic of privacy coins has been controversial among the crypto community. Some point to their ability to ensure greater anonymity during transactions, while others raise concerns about using XMR to shield illicit transactions and its alleged embracement by extremist groups. Last year, Kraken delisted XMR for its U.K. customers, citing regulatory pressure.  \cite{cointelegraph}
\end{quotation}

Beyond the obvious clarification about a good not having any kind of morality at all\footnote{In this case, Monero is not “immoral” because it's used by criminal. Immorality resides in the individual's action that threats the life and/or the private property of others. The asset \textit{per se} has no intrinsic morality.}, Monero marginalization might be because of the actual impossibility of tracking the transactions in the blockchain; and given its internal protocols of safety to avoid any kind of attack is what makes the States, unlike Bitcoin, prohibits it (it cannot be controlled whether directly or indirectly).

In order to respond the central issue of this section: legality does not implies morality. In the academic discussions about \textit{ethics}\footnote{Precisely, about the categorical imperative exposed by Immanuel Kant.} and legality, it's common to cite the example of a random individual who helps to refugee a jewish in the National-Socialist Germany of the 1940s. Helping a person, despite its religion, is a morally “good” act, but in this case it goes against the German law and the duty of the citizen. However, handing over the jewish to the German authorities it's a morally “bad” act (the person will be tortured, disappeared, shot, etcetera), although this implies the respect over laws and the duty of its role as a citizen. Despite the trivial example, this serves as a counter-example about the law compliance issue and the morally good acts that are incompatible.

In conclusion: the user must not be concern about if its actions are \textit{legal} or \textit{illegal}, but if its actions are morally \textit{good} or \textit{bad}. This distinction it can be done as long as it doesn't affect the life and/or the private property of a third party. Without going further, the State functions through violence (a morally \textit{bad} act) through restrictions on the individual freedom, being this in the economic plane (taxes, regulations, depreciation of the national currency) as well as the freedom itself (understand this as the arbitrarily restriction imposed during the COVID pandemic); even expressly threatening the lives of people (for example, the Russian invasion in Ukraine).

\section{Entrepreneurship}
Starting a business is a complex process. This starts from the initial idea of what one could offer to the market, through the image of the new business, even its marketing. Regarding to the manufacturing process or service, one must consider the raw materials and the available supplies for its execution, \textit{id est}, the initial capital for starting a business (including the \textit{material} capital and \textit{intellectual} capital). And regarding to its application, one must think, with such initial capital, in the manufacturing processes or services to create a \textit{standardization} of the work processes.

Within the \textit{legal} framework, it's difficult to establish “equal” parameter for each persons who lives around the world, because the legal requirements are different for each country. Some of them are more “flexible” to start a business; and some others require a long and tedious bureaucratic process\footnote{From the decision of starting the legal process until, finally, trade with the State permission, one can spend half day of paperwork in New Zealand \cite[p.~4]{db:newzealand}, or 230 days in Venezuela \cite[p.~4]{db:venezuela}.}, in which one has to spend a lot of money for its validation (even, being involved in acts of corruption). Given the agorist \textit{ethos}, generally speaking in the cryptocurrencies communities in which it's assumed this asset as a medium of exchange, business are created to grow \textit{parallel} markets\footnote{This means, an \textit{independent} market from the \textit{fiat}-based economy, in which cryptocurrencies are used as money. It's used the word “parallel” and not “illegal” because there's a rapid negative connotation towards the word “illegal”; but, as mentioned in the previous section, legality is not a valid parameter to establish the morality of human action.}, and overpass all of these state controls.

In this sense, trading to offer goods and services in the \textit{legal} or \textit{parallel} market it's “the same thing”. Obviously, for the manufacture of certain specific goods, there will be greater state controls and restrictions towards private actors, depending on the magnitude of the invested capital to produce it (it won't be the same cooking bread and sell it in the parallel market than fabricate an open-source car and sell it). Beyond these differences, there is no requirement of any process or special validation to start a business and sell in the parallel market: in fact, conceptually speaking, there will \textit{always} be a higher degree of freedom in this market than the \textit{legal} one.

\section{Open-Source Brands}
Open-Source BrandsL would be, in simple terms, an already created model business: both the “image” of the brand and the entire standardized process of manufacturing/service is already settled in digital documents. This kind of business model is not, by any mean, a necessary requisite to create such parallel market. In this document it is presented \textit{another} way to run a business in order to encourage crypto markets to detach from the exchange rate of the cryptocurrency in \textit{fiat}\footnote{About this, is referred as the act of buy cryptocurrencies “when the price is low” and sell it “when the price is high”. In this exchange, it's used cryptocurrencies as a financial asset to earn more money; and, despite the fact that it could contribute to a higher liquidity in the market, this surely doesn't contributes to the creation of a parallel market at all.}. Basically, it's about to offer a \textit{business model}, in an “open source way”, in exchange for a fee for each sale (this is analogous, in the actual economy, to create a patent of a good, so that a company can take such patent and fabricating on its own).

Why, then, speak of an \textit{open source brand}? Given the experience of the development of \textit{open source} software, the work methodology behind this process (public scrutiny and renewal of processes) could be, also, applied to projects \textit{outside} the digital world. Just as it has been taken Bitcoin as a standard model of cryptocurrency, and it has been made different modification of the code to launch new cryptocurrencies, the same could happen with open source brands. This means, no only one could offer brands of the same category, competing with themselves; but it can also rise new brands that offers \textit{very} similar products between themselves, competing with the price and fees offered in the market (this last point will be developed in the next paragraphs).

It's pretended that, in the long run, there could be a situation in which people around the world, that wants to start \textit{immediately} a business, can choose different brands within a “brands catalog”. And so, brands that please them, or it could be economically more viable in the location they are, can be taken and used under the standardized work processes that the same brand stipulates, to sell it in the parallel local market. This idea would suppose, immediately, a reject due to the economic incentives on the creator of the brand to be commercialized; but this will also be exposed in the next paragraphs.

It's also expected that the manufacturing and sales of these products can be valued in different cryptocurrencies (and not just in a \textit{single} cryptocurrency, although this will depend on the subjective preferences of the seller); leading to a greater liquidity in the market. And this will provoke different market opportunities for new brands and competitors to emerge, and the prices of the offered goods and services to be regulated automatically in the market.

\subsection{Advantages}
What advantages an open source brand could have in comparison of a common business? Technically speaking, an open source brand \textit{is} a business; and this will not suppose a change in the paradigm that “leaves obsolete” a common business model. Just because it doesn't have much difference with the conception of an open source software, it could be mentioned some advantages when at the moment of creating an open source brand:

\begin{enumerate}
\item Transparency in the work processes.
\item Public scrutiny of the brand.
\item Skipping State controls.
\end{enumerate}

\subsubsection{Transparency in the work processes}
Just like in the digital world, where programs are under an open-source license in which any person can have access to such code and make a scrutiny of its operations, the same it is proposed with business and its work processes. In what thing can this feature differentiate with a traditional business? The difference would be the \textit{transparency} of the work processes (because these are publicly described).

Given that the word \textit{transparency}, in the crypto environment, is referred to the visualization and tracking of the user's transactions, here is referred to the work processes. It has nothing to do with publicly show the user's account whatsoever, it only specifies about \textit{how} to work with the raw materials.

This \textit{transparency} of the work processes isn't something exclusively in the digital world, but rather also in the real world. An everyday example is in the gastronomic world, where in internet and in diaries, magazines and books are established the raw materials and the work processes to cook a specific meal. When someone checks a recipe from the book \textit{Gino's Italian Express} by the Italian cooker Gino D'Acampo, it's being working with the raw material and the cook process that has been stipulated by Gino D'Acampo himself. Instead of personally cooking the meal from the user who desires it, he leaves the entire work process in a book for the user to buy it and cook the recipe (and obtaining a fee from the book sale).

The business model of the open source brand, conceptually speaking, refers to the same thing\footnote{Another example of \textit{transparency} is when parents, when they take their son or daughter to a private school, they desire is to see the curriculum taught in that school. Depending on their subjective preferences, and the comparison with other private schools, parents could take a decision about it and they'll choose one institution over another (when the economic factor is discarded). The standardization of the teaching process is “materialized” in the curriculum: the process can be done partially or totally; but this is a guide for the parents to know \textit{a priori} what kind of content it will be taught. In the opposite case, the content will be revealed \textit{a posteriori}: they'll only know the curriculum of the day, after being taught to its son or daughter. Therefore, in this case, the \textit{transparency} is both beneficial for school and parents. The institution will make sure a monthly income in the long term, and the parents will make sure of paying for a good long-term service. Besides, there's an extra advantage in this scheme that allows competition of the curriculum among different schools.}. By detailing the work processes to elaborate certain goods or services, a third party cannot only have a higher degree of security by knowing \textit{how} it's been elaborated, but also this third party will be able to replicate this to offer the same product.

\subsubsection{Public scrutiny of the brand}
The same thing that happens with the \textit{open} software, in this case it will also be able to make a revision on the entire work process and raw materials (it's worth to mention that public scrutiny is only possible when there is \textit{transparency}). This revision could derive in cases where interested people can correct certain errors overlooked by the inventor. In the case of not correcting it, given the \textit{open source} nature of the brand, this could be an opportunity for \textit{alternatives} brands to emerge with a similar process\footnote{This is what happened, for example, with Bitcoin and the cryptocurrencies derived from the same code.}.

Public scrutiny, applied to brands that offers non-digital goods, can serve to replicate the work processes and publicly object if it's feasible or not (and if the produced good is \textit{legitimate} or not). Also, it could also serve in case where the inventor has designed and created a good, but he hasn't the necessary capital to manufacturing it. The most simple example, in order to understand this idea, could be about the 3D printing market, when, maybe, a person has designed a manual machine, but he can't buy a 3D printer due to its high cost (and its lower purchasing power to acquire it). Given the fact that is an open source brand of 3D printings, being a complex machine, it could be left to a public scrutiny for the model to be analyzed, replicated and made the necessary revisions in the case where certain flaws may appear\footnote{This is similar to having a company in which the new designs are sent to the R\&D Department.} (and that people know whether this design works or not). Once the design is validated, the brand could be used to fabricate such machine.

This concept of the public scrutiny, also, can be of a great help when one must fabricate a good which requires a huge initial capital but, for example, in the location where the inventor lives, given a greater degree of authoritarianism and State control, it's not possible for him or her to fabricate it (both legally and economically). Here, the quality control of the design could be done \textit{outside} the State surveillance (this means, in “freer” locations) to let the inventor know if its project is free of flaws and if it is feasible its invention.

For example, a person located in Equatorial Guinea, who is interested in creating an open source brand of vehicles (with a specific logo and a standardized fabrication process), digitally designs a car. Her problem lies in not been able to materialize it because of her lack of resources and a huge amount of legal obstacles in order to fabricate it and testing it. By publishing her design and stipulating it as an open source design, an Argentinian could fabricate it and check it for a quality control. A Mexican could check this quality control and arriving to the conclusion that this data is inconsistent, being able to be done a minimal change in the design in order to be more efficient. Given the public scrutiny of this design to improve it, the Ecuatoguinean could make the final design in a new version; and she would publish it for its manufacturing (in exchange, of course, of a fee for each sale of the fabricated car).

\begin{center}
*****[NOT FULLY TRANSLATED YET]*****
\end{center}

\newpage

\section{Flow chart}

\begin{center}
\tikzstyle{block-w} = 
	[rectangle, draw, fill=white, text centered, 
	rounded corners, text width=6.0em, node distance=2cm]
\tikzstyle{block-y} = 
	[rectangle, draw, fill=yellow!75, text centered, 
	rounded corners, text width=6.0em, node distance=2cm]
\tikzstyle{block-g} = 
	[rectangle, draw, fill=green!50, text centered, 
	rounded corners, text width=6.0em, node distance=2cm]
\tikzstyle{block-r} = 
	[rectangle, draw, fill=red!50, text centered, 
	rounded corners, text width=4.5em, node distance=2cm]
\tikzstyle{if} = 
	[diamond, draw, fill=orange!50, text centered, 
	rounded corners, text width=2.5em, node distance=2cm, minimum height=2.5em]
\tikzstyle{circle} = 
	[ellipse, draw, fill=blue!25, text centered, 
	node distance=2cm, minimum height=3em]  
\tikzstyle{line} = [draw, -latex']

\begin{tikzpicture}[node distance = 1.8cm, auto]
	\node [block-w] (cliente) {Cliente};
	\node [block-w, below of=cliente] (app) {Aplicación};
	\node [circle, below of=app] (qr) {Código QR};
	\node [block-w, below left of=qr] (inv) {Inventor};
	\node [block-w, below right of=qr] (vend) {Vendedor};
	\node [block-y, below of=vend] (gpg1) {$ GPG_{1} $};
	\node [block-y, below of=inv] (gpg2) {$ GPG_{2} $};
	\node [block-y, below of=gpg1] (dir1) {Dirección y porcentaje de comisión};
	\node [block-y, below of=gpg2] (dir2) {Dirección y porcentaje de comisión};
	\node [circle, below of=dir1] (comp) {Comprobación};
	\node [if, below of=comp] (if) {Si};
	\node [block-r, below left of=if] (not-ok) {Inválido};
	\node [block-r, left=4em of not-ok] (error) {«Error»};
	\node [block-g, below right of=if] (ok) {Válido};
	\node [circle, below of=ok] (pago) {Pago};
	\path [line] (cliente) -- (app);
	\path [line] (app) -- (qr);
	\path [line] (qr) -- (inv.north);
	\path [line] (qr) -- (vend.north);
	\path [line] (vend) -- (gpg1);
	\path [line] (inv) -- (gpg2);
	\path [line] (gpg1) -- (dir1);
	\path [line] (gpg2) -- (dir2);
	\path [line] (dir1) -- (comp);
	\path [line] (dir2) |- (comp);
	\path [line] (comp) -- (if);
	\path [line] (if) -- (ok);
	\path [line] (if) -- (not-ok);
	\path [line] (not-ok) -- (error);
	\path [line, dashed] (error) |- (app.west);
	\path [line] (ok) -- (pago);
	
\end{tikzpicture}
	
	
	
	
	

\end{center}

Valid $ \rightarrow ADDRESS_{GPG} = ADDRESS_{QR} \wedge \%_{GPG} = \%_{QR} $

Invalid $ \rightarrow ADDRESS_{GPG} \neq ADDRESS_{QR} \vee \%_{GPG} \neq \%_{QR} $

\bibliographystyle{apacite}

\bibliography{osbbiblio.bib}

\end{document}